\chapter{Archivos de configuración para la instalación} \label{chap:manualdeinstalacion}

\section{Archivo de configuración del Servidor 1}\label{sec:ConfServer1}
En esta sección del apéndice mostramos el archivo de configuración \textit{local.conf} que contiene la configuración necesaria para la instalación de openStack en el Servidor 1 junto con la descripción de los mismos.

\begin{itemize}
\item Archivo de configuración \textit{local.conf} para el Servidor 1:
\end{itemize}

\begin{lstlisting}[style=Consola]
stack@jorge-All-Series:~/devstack$ vi local.conf

[[local|localrc]]
###########################################################
# Configurando el entorno a crear
###########################################################
PUBLIC_INTERFACE=enp3s0
HOST_IP=10.5.1.201
FLOATING_RANGE=10.5.1.224/24
FLAT_INTERFACE=enp3s0:1
Q_FLOATING_ALLOCATION_POOL=start=10.5.1.224,end=10.5.1.254

# Claves de acceso para los distintos servicios
ADMIN_PASSWORD=temporary
MYSQL_PASSWORD=temporary
RABBIT_PASSWORD=temporary
SERVICE_PASSWORD=$ADMIN_PASSWORD
SERVICE_TOKEN=temporary

# Configuramos el primer servidor como la primera region
REGION_NAME=RegionOne

############################################################
# Customizando entorno
############################################################
# Permitir el uso de Pip para manejo de paquetes
PIP_USE_MIRRORS=False
USE_GET_PIP=1

#OFFLINE=False
# Reclone nos permite reconstruir los escenarios creados en caso de reinicio del servidor
RECLONE=True

# Habilitar log para reportes de fallos y otros.
LOGFILE=$DEST/logs/stack.sh.log
VERBOSE=True
ENABLE_DEBUG_LOG_LEVEL=True
ENABLE_VERBOSE_LOG_LEVEL=True

# Neutron ML2 with OpenVSwitch
Q_PLUGIN=ml2
Q_AGENT=openvswitch

# Habilitar grupos de seguridad
Q_USE_SECGROUP=False
LIBVIRT_FIREWALL_DRIVER=nova.virt.firewall.NoopFirewallDriver

# Habilitar Heat
enable_plugin heat https://git.openstack.org/openstack/heat stable/queens
# Habilitar Heat para su uso desde el Dashboard
enable_plugin heat-dashboard https://git.openstack.org/openstack/heat-dashboard stable/queens
# Habilitar funciones de red
enable_plugin networking-sfc git://git.openstack.org/openstack/networking-sfc stable/queens

# Descarga de automatica de imagenes para el servicio de Heat.
IMAGE_URL_SITE="http://download.cirros-cloud.net"
IMAGE_URL_PATH="/0.4.0/"
IMAGE_URL_FILE="cirros-0.4.0-x86_64-disk.img"
IMAGE_URLS+=","$IMAGE_URL_SITE$IMAGE_URL_PATH$IMAGE_URL_FILE

# Habilitar Ceilometer
enable_plugin ceilometer https://git.openstack.org/openstack/ceilometer stable/queens
# Habilitar Tacker
enable_plugin tacker https://git.openstack.org/openstack/tacker stable/queens
# Habilitar Tacker para su uso desde Horizon
https://git.openstack.org/cgit/openstack/tacker-horizon stable/queens

# Servicios adicionales para Nova
enable_service n-novnc
enable_service n-cauth
disable_service tempest
[[post-config|/etc/neutron/dhcp_agent.ini]]
[DEFAULT]
enable_isolated_metadata = True
\end{lstlisting}

\begin{itemize}
\item Descripción de los parámetros principales del archivo de configuración \textit{local.conf} no descritos en el mismo:

\begin{itemize}
\item Cabecera del archivo, es obligatoria. [[local|localrc]]. 
\item PUBLIC\_INTERFACE=enp3s0. Intefaz de red física que servirá como gateway hacia la red pública.
\item HOST\_IP=10.5.1.201. Dirección IP del host que alojará OpenStack.
\item FLOATING\_RANGE=10.5.1.224/24. Rango de direcciones IPs flotantes.
\item FLAT\_INTERFACE=enp3s0:1. Interfaz de red que conecta el entorno creado a nuestra máquina.
\item Q\_FLOATING\_ALLOCATION\_POOL=start=10.5.1.224,end=10.5.1.254. A pesar de haber creado un rago de IPs flotantes, podemos establecer únicamente una cantidad de direcciones en ese rango de las que disponer.
\item Claves de acceso para los distintos servicios. Después de estos parámetros, en el arhcivo se usan distintas contraseñas: una para el administrador del entorno cloud creado con acceso a la conguracion completa, otra para la base de
datos que se especifica que sea MySQL, otra para la cola de mensajes RabbitMQ incorporada por defecto en este paquete  y por ultimo la contraseña de servicio predeterminada para la autenitación en Keystone de los servicios.
\end{itemize}
\end{itemize}

El resto de parámetros se describen en el propio archivo. Como se puede ver se especifica que este Servidor haga las veces de la región One.

\section{Archivo de configuración del Servidor 2}\label{sec:ConfServer2}
En el caso del Servidor 2 la configuración es la que sigue:

\begin{lstlisting}[style=Consola]
[[local|localrc]]
############################################################
# Configurando el entorno a crear
############################################################
PUBLIC_INTERFACE=enp0s31f6
HOST_IP=10.5.3.201
FLOATING_RANGE=10.5.3.224/24
FLAT_INTERFACE=enp0s31f6:2
Q_FLOATING_ALLOCATION_POOL=start=10.5.3.224,end=10.5.3.254

ADMIN_PASSWORD=temporary
MYSQL_PASSWORD=temporary
RABBIT_PASSWORD=temporary
SERVICE_PASSWORD=$ADMIN_PASSWORD
SERVICE_TOKEN=temporary

# Para primer escenario, segunda region.
disable_service horizon
KEYSTONE_SERVICE_HOST=10.5.1.201
KEYSTONE_AUTH_HOST=10.5.1.201
REGION_NAME=RegionTwo
KEYSTONE_REGION_NAME=RegionOne
\end{lstlisting}

Los parámetros de configuración del entorno en este caso se adaptan a las interfaces de red del Servidor 2. Además se indicaa que se trata de la región Two y que el servicio de autenticación estará al cargo de la región One de la cual indicamos también su IP.