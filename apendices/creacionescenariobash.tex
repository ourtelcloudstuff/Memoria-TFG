\chapter{Creación de un escenario mediante scripting en bash} \label{chap:creacionescenariobash}

Para la automación del escenario, crear el script \textit{cli\_creacion\_escenario.sh} que sigue y ejecutarlo:

\begin{lstlisting}[style=Consola]
source openrc admin admin
openstack project create --description 'Escenario creado con cli' cli-project --domain default
openstack quota set cli-project --instances 10
openstack quota set cli-project --ram 4096 
openstack user create --project cli-project --password temporary cli-user
openstack role add --user cli-user --project cli-project Member
source openrc cli-user cli-project
openstack network create cli-private-net
openstack subnet create --network cli-private-net --subnet-range 192.168.20.0/24 cli-private-subnet
openstack router create cli-router
openstack router set cli-router --external-gateway public
openstack router add subnet cli-router cli-private-subnet
openstack floating ip create public
openstack floating ip create public
openstack keypair create cli-keypair > cli-keypair.pem
chmod 0600 cli-keypair.pem
ssh-add cli-keypair.pem
openstack security group create cli-secgroup
openstack  security group rule create --remote-ip 0.0.0.0/0 --protocol tcp --dst-port 22  --ingress cli-secgroup
openstack  security group rule create --remote-ip 0.0.0.0/0 --protocol icmp --ingress cli-secgroup
openstack  security group rule create --remote-ip 0.0.0.0/0 --protocol tcp --dst-port 80:80  --ingress cli-secgroup
openstack image create --disk-format qcow2 --min-disk 2 --file /opt/stack/devstack/images/cirros-0.4.0-x86_64-disk.img  --private cli-cirros
openstack flavor create --ram 512 --disk 4 --vcpus 1 m1.little
nova boot --flavor m1.little --image cli-cirros --key-name cli-keypair --security-groups cli-secgroup --nic net-name=cli-private-subnet cli-instance-1
openstack volume create --image cli-cirros \
  --size 8 --availability-zone nova cli-cirros-volume
openstack server add volume cli-instance-1 \
  cli-cirros-colume --device /dev/vdb
\end{lstlisting}