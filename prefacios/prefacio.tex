\chapter*{}
%\thispagestyle{empty}
%\cleardoublepage

%\thispagestyle{empty}

\begin{titlepage}
 
 
\setlength{\centeroffset}{-0.5\oddsidemargin}
\addtolength{\centeroffset}{0.5\evensidemargin}
\thispagestyle{empty}

\noindent\hspace*{\centeroffset}\begin{minipage}{\textwidth}

\centering
%\includegraphics[width=0.9\textwidth]{imagenes/logo_ugr.jpg}\\[1.4cm]

\textsc{ \Large TRABAJO FIN DE GRADO\\[0.2cm]}
\textsc{ GRADO EN INGENIERÍA DE TECNOLOGÍAS DE TELECOMUNICACIÓN}\\[1cm]
% Upper part of the page
% 

 \vspace{0.3cm}

%si el proyecto tiene logo poner aquí
%\includegraphics{imagenes/portada/openstack_logo.png} 
 \vspace{0.5cm}

% Title

{\Huge\bfseries Diseño y despliegue de Funciones de Red Virtuales (VNFs) utilizando OpenStack\\
}
\noindent\rule[-1ex]{\textwidth}{3pt}\\[3.5ex]
%{\large\bfseries \\[4cm]}
%{\large\bfseries}
%\end{minipage}

\vspace{0.1cm}
%\noindent\hspace*{\centeroffset}\begin{minipage}{\textwidth}
\centering

\textbf{Autor}\\ {Alejandro Toledo Juan}\\[2.5ex]
\textbf{Director}\\
{Jorge Navarro Ortiz}\\[1.5cm]
%\includegraphics[width=0.15\textwidth]{imagenes/portada/tstc.png}\\[0.1cm]
\textsc{Departamento de Teoría de la Señal, Telemática y Comunicaciones}\\
\textsc{---}\\
Granada, mes de 201
\end{minipage}
%\addtolength{\textwidth}{\centeroffset}
%\vspace{\stretch{2}}

 
\end{titlepage}






\cleardoublepage
\thispagestyle{empty}

\begin{center}
{\large\bfseries Diseño y despliegue de Funciones de Red Virtuales (VNFs) utilizando OpenStack}\\
\end{center}
\begin{center}
Alejandro Toledo Juan\\
\end{center}

%\vspace{0.7cm}
\noindent{\textbf{Palabras clave}: Cloud, Devstack, Heat, IaaS, IT, NFVs, OpenStack, Tacker, VIM, VNF}\\

\vspace{0.7cm}
\noindent{\textbf{Resumen}}\\

\jorge{Yo introduciría el concepto de nube y su relevancia actual, metiendo algún párrafo que otro. Y después sí comentas qué se hace en este trabajo (los párrafos que has puesto debajo).}
\jorge{Por cierto, puedes usar \textbackslash alejandro\{texto\} para poner comentarios tuyos. Así los distingo rápido (están en verde, los míos en violeta, y los que sean muy importantes en rojo). Te pongo un ejemplo debajo:}
\alejandro{Esto es un ejemplo.}

En este trabajo se pretende realizar la instalación y configuración de un \textit{cloud} basado en el software OpenStack, con el objetivo de ejecutar en el mismo una serie de funciones de red virtualizadas (VNFs). Estas funciones se deberán poder instanciar y gestionar de forma automática.

\jorge{Los términos en inglés ponlos en cursiva, con \textbackslash textit\{texto\}}

Este trabajo busca una visión para profesionales de IT que quieren una descripción general de alto nivel de OpenStack, y que desean saber si OpenStack es la solución adecuada para satisfacer las necesidades de IT de su organización. En él también ayudaremos a cualquier persona que quiera establecer un entorno de prueba OpenStack a pequeña escala a adquirir experiencia trabajando con OpenStack.

\cleardoublepage


\thispagestyle{empty}


\begin{center}
{\large\bfseries Project Title: Design and Dimensioning Virtual Network functions (NFVs) using OpenStack}\\
\end{center}
\begin{center}
Alejandro Toledo Juan\\
\end{center}

%\vspace{0.7cm}
\noindent{\textbf{Keywords}: Cloud, Devstack, Heat, IaaS, IT, NFVs, OpenStack, Tacker, VIM, VNF}\\

\vspace{0.7cm}
\noindent{\textbf{Abstract}}\\

\jorge{Idem, actualiza el resumen en inglés cuando hayas cambiado el resumen en español.}

In this work we intend to perform the installation and configuration of a cloud based on the OpenStack software, in order to execute a series of virtualized network functions (VNFs). These functions should be able to instantiate and manage automatically.

This work seeks a vision for IT professionals who want an
OpenStack general-level description, and know if OpenStack
is the solution for the right one to meet the IT needs of their organization. At the same time we will help anyone
to deploy an OpenStack  small scale test environment to gain experience working with OpenStack.




\chapter*{}
\thispagestyle{empty}

\noindent\rule[-1ex]{\textwidth}{2pt}\\[4.5ex]

Yo, \textbf{Alejandro Toledo Juan}, alumno de la titulación GRADO EN INGENIERÍA DE TECNOLOGÍAS DE TELECOMUNICACIÓN  de la \textbf{Escuela Técnica Superior
de Ingenierías Informática y de Telecomunicación de la Universidad de Granada}, con DNI 71223946F, autorizo la ubicación de la siguiente copia de mi Trabajo Fin de Grado en la biblioteca del centro para que pueda ser consultada por las personas que lo deseen.

\vspace{6cm}

\noindent Fdo: Alejandro Toledo Juan

\vspace{2cm}

\begin{flushright}
Granada a día de mes de 201 .
\end{flushright}


\chapter*{}
\thispagestyle{empty}

\noindent\rule[-1ex]{\textwidth}{2pt}\\[4.5ex]

D. \textbf{Jorge Navarro Ortiz}, Profesor del Área de Ingeniería Telemática del Departamento de Teoría de la Señal, Telemática y Comunicaciones de la Universidad de Granada.

\vspace{0.5cm}


\vspace{0.5cm}

\textbf{Informan:}

\vspace{0.5cm}

Que el presente trabajo, titulado \textit{\textbf{ Gestión de Funciones de Red Virtuales (NFVs) utilizando OpenStack}}, ha sido realizado bajo su supervisión por \textbf{Alejandro Toledo Juan}, y autorizo la defensa de dicho trabajo ante el tribunal que corresponda.

\vspace{0.5cm}

Y para que conste, expiden y firman el presente informe en Granada a X de mes de 201 .

\vspace{1cm}

\textbf{El director:}

\vspace{5cm}

\noindent \textbf{Jorge Navarro Ortiz \ \ \ \ \ }

\chapter*{Agradecimientos}
\thispagestyle{empty}

       \vspace{1cm}


Poner aquí agradecimientos...

